\section{Übersicht App-Technologien}
Um Apps zu entwickeln gibt es viele Möglichkeiten.
Diese lassen sich grob in folgende Arten einteilen

\begin{table}[htbp]
    \begin{tabular}{|l|l|}
    \hline
    Art      & Charakteristik                                                                                                                                                       \\ \hline
    Hybrid   & Web-Apps werden im nativen Kontext in einer WebView eingebunden                                                                                                      \\ \hline
    Native   & \begin{tabular}[c]{@{}l@{}}Adressieren konkrete Zielplattformen und deren Programmiersprachen.\\ Android (Java, Kotlin, Dart), iOS (Objective-C, Swift)\end{tabular} \\ \hline
    Web Apps & Über einen Server bereitgestellte plattformunabhängige Anwendungen                                                                                                   \\ \hline
    \end{tabular}
    \caption{Übersicht Arten von Apps}
\end{table}

Ich bin großer Fan von \ac{WORA} und entwickle üblicherweise vor allem Web-Apps.
Um etwas neues zu lernen und daran zu wachsen, entschied ich mich,
dieses Mal dazu eine Cross-Plattform-Technologie zu nutzen.
Die Entscheidung fiel dabei auf das von Google entwickelte Open Source \ac{UI}-Kit \textit{Flutter}.

\begin{figure}[H]
    \minipage[t]{0.4\textwidth}
        \includegraphics[width=\linewidth]{flutter_logo}
        \caption{ Flutter-Logo }
    \endminipage\hfill
    \minipage[t]{0.4\textwidth}
        \includegraphics[width=\linewidth]{dart_logo}
        \caption{ Dart-Logo }
    \endminipage\hfill
\end{figure}
