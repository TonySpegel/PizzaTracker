\section{Motivation und Zielsetzung}
\subsection{Motivation}

Diese Hausarbeit dokumentiert die Entwicklung eines Prototyps zur
vereinfachten Findung von unbelegten Räumen der \ac{EAH} in Anbetracht
von Usability-Kriterien. Die Motivation diese
Anwendung zu programmieren, entstand zunächst aus dem Bedarf heraus,
möglichst schnell Räume zu finden, die für längere Zeit unbelegt sind.
Dies ist besonders zur Prüfungszeit wichtig, da man des Öfteren allein oder
in der Gruppe ungestört lernen möchte. Die Bibliothek ist kein
geeigneter Ort um sich zu unterhalten, zwischen den Gängen
gibt es keine Tafeln und direkt auf dem Campus ist es je nach Witterung
auch nicht optimal, um dort zu lernen. Optimal sind demnach nur die Räume
der EAH. Durch Diskussionen mit anderen Studierenden stellte sich heraus,
dass es besonders wichtig ist, per Smartphone so viele freie Räume
wie möglich angezeigt zu bekommen.
Um herauszufinden, welche Räume belegt sind und welche nicht,
gibt es drei Möglichkeiten:

\begin{itemize}
    \itemsep-0.4em
    \item Räume besuchen und hoffen, dass diese nicht belegt sind
    \item Raumbelegungsplan der EAH-Website nutzen
    \item "Freie Räume" der der EAH-Website nutzen
\end{itemize}

Räume einen nach dem anderen zu öffnen und zu hoffen dass diese
nicht belegt sind ist sicherlich eine der direktesten aber auch der zeitintensivsten
Möglichkeiten um zum gewollten Ziel zu gelangen. Deutlich bequemer klingen zunächst
die beiden anderen genannten Möglichkeiten. Die EAH bietet auf der Website\\
\url{http://stundenplanung.eah-jena.de/raeume/} die Möglichkeit, Raum für Raum
dessen Belegung zu prüfen. Aber auch dies ist sehr aufwendig weil ein direkter
Vergleich nicht möglich ist. Der letzte genannte Punkt, zu finden unter
\url{http://stundenplanung.eah-jena.de/raumsuche/} kann als eigentlicher
Anstoß für die Entwicklung dieser Anwendung gesehen werden.
Zu Beginn der Entwicklung war diese Funktion nicht verfügbar. Dies wurde zum Anlass genommen,
eine Lösung zu entwickeln, die insbesondere unter Aspekten der Usability und \ac{UX} Verbesserungen
bieten soll. Ist diese Funktion verfügbar, so bietet diese zwar grundsätzlich das gewünschte Ergebnis,
ist aber unter Usability/UX-Aspekten als unzureichend anzusehen.

\newpage

\subsection{Zielsetzung}

Eines der Hauptziele dieser neu zu entwickelnden Lösung soll es sein,
dem Nutzer so schnell wie möglich - und dies ohne Benutzereingaben,
verfügbare Räume zu präsentieren. Jeder soll diese Anwendung ohne
Installation einer App nutzen können.
Um die Komplexität der Anwendung gering zu halten, verzichtet
diese auf die Verwendung einer Datenbank, auch um das Hosting
der Anwendung zu erleichtern. Folgende Hauptziele wurden hierbei definiert:
\begin{itemize}
    \itemsep-0.4em
    \item Entwicklung einer plattformunabhängigen Web-App
    \item Keine Datenbank
    \item "Einfache Benutzung" ohne Benutzereingaben
    \item Filterung der Ergebnisse und Individualisierbarkeit
\end{itemize}

