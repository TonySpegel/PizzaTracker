\section{Motivation}

Diese Hausarbeit dokumentiert die Entwicklung eines Prototyps zur
vereinfachten Findung von unbelegten Räumen der \ac{EAH} in Anbetracht
von Usability-Kriterien. Die Motivation diese
Anwendung zu programmieren, entstand zunächst aus dem Bedarf heraus,
möglichst schnell Räume zu finden, die für längere Zeit unbelegt sind.
Dies ist besonders zur Prüfungszeit wichtig, da man des Öfteren allein oder
in der Gruppe ungestört lernen möchte. Die Bibliothek ist kein
geeigneter Ort um sich zu unterhalten, zwischen den Gängen
gibt es keine Tafeln und direkt auf dem Campus ist es je nach Witterung
auch nicht optimal, um dort zu lernen. Optimal sind demnach nur die Räume
der EAH. Durch Diskussionen mit anderen Studierenden stellte sich heraus,
dass es besonders wichtig ist, per Smartphone so viele freie Räume
wie möglich angezeigt zu bekommen.
Um herauszufinden, welche Räume belegt sind und welche nicht,
