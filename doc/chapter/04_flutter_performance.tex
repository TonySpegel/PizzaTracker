\section{Flutter's Performance}
Eine der Besonderheiten von \textit{Flutter} ist es, dass dieses UI-Kit weder eine WebView
noch die vom Betriebssystem mitgelieferten Widgets benutzt. Widgets sind in der
Welt von \textit{Flutter} alles von Bedienelemente bis hin zu Layout-Helfern.
Statt diese mitgelieferten Widgets zu nutzen, setzt \textit{Flutter} auf eine
eigene Rendering-Engine welche häufig mit einer 2D-Spiele-Engine verglichen wird.
Eines der Entwicklungsziele von \textit{Flutter} war es nämlich,
besonders performante Apps entwickeln zu können welche mit einer hohen
Hertz-Zahl (60-120 \textit{Hz}) laufen. Mit diesem Ansatz ist es möglich,
die gesamte UI über den Grafikchip des Systems zu berechnen und die CPU zu entlassen.
Dadurch wird es außerdem möglich, Änderungen die man im Code vornimmt,
nahezu ohne Verzögerung in der App zu sehen, ohne das ein Neuladen oder ähnliches
notwendig ist.
