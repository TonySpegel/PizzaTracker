\section{Fazit}

Grundsätzlich kann man sagen, dass die anfangs gesteckten Ziele alle erfüllt wurden.
Die Entwicklung einer plattformunabhängigen Web-App sowie der verzicht eine Datenbank zu nutzen,
waren quasi von Anfang an gegeben. Die "Einfache Benutzung" ohne Benutzereingaben war
die eigentliche Herausforderung, welche aber durch einfache Annahmen und Tests mit Nutzern
gut umgesetzt werden konnten. Suchergebnisse zu filtern brachte mehr Probleme als
angenommen mit sich, nämlich Formularvalidierung und ein Redesign des Formulars.
Die Individualisierbarkeit blieb bisher auf der Strecke und meinte vor allem die Möglichkeit,
Räume als Favorit hinzuzufügen und diese in der Liste weiter oben oder ausschließlich
freie (oder belegte) Räume anzuzeigen. Erst im Laufe des Projektes konnte Wissen aufgebaut werden,
um die Möglichkeiten einer \ac{SPA} gewinnbringend einsetzen zu können.
Solche Learnings nachträglich einzusetzen kann schwierig sein
und wird oft auch als \textit{"Technical Debt"} bezeichnet.
Durch die neuen Möglichkeiten einer SPA ist es auch denkbar Offline-Support,
Push-\\Benachrichtigungen sowie eine tiefere Integration in das Betriebssystem des Nutzers
zu ermöglichen ohne eine Installation zur Pflicht zu machen. Der Schlüssel für
diese Features heißt \textit{"Service-Worker"} und beschreibt eine JavaScript-Funktionalität
um clientseitige Proxy-Dienste anzubieten. Solche Dienste sind in der Lage Netzwerkanfragen
abzufangen und zu manipulieren. Nichtsdestotrotz war und ist dies ein Projekt welches mir
und einigen anderen Studierenden schon oft geholfen hat und weiterhin hilft.
Der Aufwand des gesamten Projekts und der damit verbundenen Wissensaneignung
beträgt in etwa zwei bis drei Monate.
Usability und User Experience sind sehr wichtige Faktoren zum Erfolg
einer Anwendung - denn niemand will etwas nutzen das vielleicht alles kann aber unbenutzbar ist.
